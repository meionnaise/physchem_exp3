A short introductory section incorporating the experimental aims. You may include figures and diagrams, but
these should be prepared yourself or properly attributed.


- first part uses emission spectroscopy to measure rate of electron transfer from phoot excited transition metal complex to another species. In this case, the competition between light emissio  and electron transfer enabkes a sensitive measurement of the electron transfer rate, given knowledge of je rate of relaxation of tje excited state in the absence of the electron transfer partner

in second aprt, use absorption spectrophotomeyry to determine the empirical formula ajdn equilibrium stability constant of a transition metal complex based on changes to hte absorption spectrum as a function of hte concentrations of te chemical components of the complex that are mixed together

aims: top use spectroscopy nad spectrophotometry to measure rxn rate coefficients and equilibrium constants of physiochemcial processes
- understand relationship between absorption., emission and excitation specra of a molecule
understand what a stern-volmer plit is and hw it can be used to measure rxn rate coefficients from measurements of luminescence quenching
distinguish between diffusion limited and rxn limited processes

isosbestic point n figure ut whsat it means
\textbf{from here is legit}
The experiment consisted of two parts in order to investigate the kinetics and thermodynamics of chemical processes, using spectroscopy and spectophotometry.

In the presence of a quenching species, luminescent molecules 

The Franck-Condon principle states that the vertical transition to occur when the absorption of light results in an electronic excited state with the same spin multiplicity \autocite{frank_condon}.

Part 1 investigated the electron transfer rate from luminescent quenching of \ce{Ru(bpy)_{3}^{2+}} by \ce{Fe^{3+} and \ce{Cu^{2+}}}. As per the lab manual\autocite{lab_manual}, the excited state of a luminescent molecule can be deactivated by the presence of a quencher molecule which inhibits emission of light. The absorption of light 


half page, 3-5 paras, talk about absorption vs excitation vs emission, go into frank condon principle, quenching, electron transfer and rate (put figs)