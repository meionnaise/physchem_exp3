\textbf{Question 1}
\begin{figure}[H]
     \centering
     \begin{subfigure}[b]{0.7\textwidth}
         \centering
         \includegraphics[width=\textwidth]{part1_q1_Fe.png}
         \caption{RENAME}
         \label{fig:part2_q1_a}
     \end{subfigure}
     \hfill
     \begin{subfigure}[b]{0.7\textwidth}
         \centering
         \includegraphics[width=\textwidth]{part1_q1_Cu.png}
         \caption{REDO}
         \label{fig:part2_q1_b}
     \end{subfigure}
     \caption{sjdfsdn}
     \label{fig:part2q1}
\end{figure}
\subsubsection*{In Class Analysis}