\begin{enumerate}
    \item Analysis was conducted at the absorbance maximum of the \ce{FUCK} complex as larger absorbance values allow for more reliable comparison of data. As shown in Figure \ref{fig:part2_q1_a}, the absorbance maximum for varying ratios remains the same, despite the shape of the curve changing with ratio. Thus, the wavelength at which peak absorption occurs is the most appropriate as this peak is common across all solution ratios.
    \item this is fucking jail
    \item wavelength at which isosbestic point occurs is roughly extinction coefficient multiplied by absorbance max (explain but also keep length of cuvette oin mind)
\end{enumerate}
% 1. Why is the absorption maximum of the complex at 530 nm an appropriate wavelength at which to
% do the analysis to determine the empirical formula and stability constant of the complex?

% 2. Compare the values of the molar extinction coefficient and stability constant of the complex
% calculated using the two different calculation methods. Discuss any similarities or differences.
% Comment on the relative accuracy of the two methods.

% 3. Determine the relationship between the molar extinction coefficient of Fe3+ and Fe3+(sal – ) at
% the isosbestic point in the measurements of different mixture ratios. (Hint: Because the stability
% constant K is very large, you can assume that the complex formation reaction essentially goes to
% completion. You should also use the fact that the total added concentration of Fe3+ and sal – is the
% same in all samples.