\begin{enumerate}
%
    \item 
\end{enumerate}

% 1. Describe and explain the similarities and/or differences between the absorption, emission, and
% excitation spectra of Ru(bpy)32+∗. (Hint: Use Figure 2.)

% 2. Normalise the emission spectra to the maximum emission intensity for each Fe3+ and Cu2+ concen-
% tration in the luminescence quenching measurements. What happens to the shape of the spectra as the Fe3+ or Cu2+ concentration increases? Why? (Hint: Plot the absorption spectra of one of the Fe3+-containing samples and one of the Cu2+-containing samples and compare them to the
% absorption spectrum of the pure Rubpy2+ sample).

% 3. Compare the quenching rate coefficients obtained from the Stern–Volmer plots with the electron-
% transfer rate coefficients calculated from Marcus theory and comment on any similarities or differ-
% ences. Note that the rate coefficient for diffusion-controlled collisions between electron-transfer
% partners under the conditions studied is ∼3.3 × 109 M−1 s−1.

% 4. Compare the rate coefficients for Fe3+ and Cu2+ (both experimental and calculated) and explain the
% similarities or differences.
