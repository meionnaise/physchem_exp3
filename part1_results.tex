SEE APPENDIX MEI REF LATER
structure discussion in paragraphs

\begin{figure}[H]
    \centering
    \includegraphics[width = 0.7\linewidth]{part1_noq.png}
    \caption{REDO TITLE using NAME OF DEVICES}
    \label{fig:part1_noq}
\end{figure}
\begin{figure}[H]
     \centering
     \begin{subfigure}[b]{0.49\textwidth}
         \centering
         \includegraphics[width=\textwidth]{part1_fe.png}
         \caption{RENAME}
         \label{fig:part1_fe}
     \end{subfigure}
     \hfill
     \begin{subfigure}[b]{0.49\textwidth}
         \centering
         \includegraphics[width=\textwidth]{part1_cu.png}
         \caption{REDO}
         \label{fig:part1_Cu}
     \end{subfigure}
     \caption{sjdfsdn}
     \label{fig:part1_fecu_spectra}
\end{figure}
Figure \ref{fig:part1_fecu_spectra} shows that for both \ce{Fe^{3+}} and \ce{Cu^{2+}}, as concentration increases, the wavelength at which peak emission intensity occurs does not change, but intensity increases with concentration.
\\
\textbf{Question 1}
THESE CONCS ARE BACKWARDS
\begin{figure}[H]
     \centering
     \begin{subfigure}[b]{0.49\textwidth}
         \centering
         \includegraphics[width=\textwidth]{part1_q1_Fe.png}
         \caption{RENAME}
         \label{fig:part1_q1_fe}
     \end{subfigure}
     \hfill
     \begin{subfigure}[b]{0.49\textwidth}
         \centering
         \includegraphics[width=\textwidth]{part1_q1_Cu.png}
         \caption{REDO}
         \label{fig:part1_q1_Cu}
     \end{subfigure}
     \caption{sjdfsdn}
     \label{fig:stern_volmers}
\end{figure}
\par Figure \ref{fig:stern_volmers} represents the following curve (eqn (6) from lab manual\autocite{lab_manual}):
\begin{equation*}
    \frac{I_0}{I} = 1 + \frac{k_q}{k_d}[Q]
\end{equation*}

So $\frac{k_q}{k_d}$ is the slope of each curve, which are:
\begin{equation*}
    \mathcolorbox{Lavender}{\frac{dy}{dx}_{\text{Fe}^{3+}} = \input{part1_q1_fe_slope.txt} \text{ @ } \input{part1_q1_fe_wl.txt} \text{ nm} }
\end{equation*}
\begin{equation*}
    \mathcolorbox{Lavender}{\frac{dy}{dx}_{\text{Cu}^{2+}} = \input{part1_q1_cu_slope.txt} \text{ @ } \input{part1_q1_cu_wl.txt} \text{ nm} }
\end{equation*}
\\
\textbf{Question 2}
\par From the lab manual\autocite{lab_manual}, $k_d = 1.7 \times 10^6\text {s}^{-1}$. Thus,
\begin{equation}
    k_q = \text{slope} \times k_d
    \label{eq:p1q2}
\end{equation}
Substituting slopes to equation (\ref{eq:p1q2}):
\begin{equation*}
    \mathcolorbox{Lavender}{k_{q,\text{Fe}^{3+}} = \input{part1_q2_fe.txt} \text{ s}^{-1}}
\end{equation*}
\begin{equation*}
    \mathcolorbox{Lavender}{k_{q,\text{Cu}^{2+}} = \input{part1_q2_cu.txt} \text{ s}^{-1}}
\end{equation*}
\\
\textbf{Question 3}
\par From tables 2 and 3 in the lab manual\autocite{lab_manual}:
   $ k_{11} = 1 \times 10 ^ 8 \text{ M}^{-1}\text{s}^{-1}$, 
    $k_{22, \text{Fe}^{3+}} = 4 \text{ M}^{-1}\text{s}^{-1}$, 
    $k_{22, \text{Cu}^{2+}} = 1 \times 10 ^ {-5} \text{ M}^{-1}\text{s}^{-1}$. And standard reduction potentials: $E^{\circ}_{\text{R}^{3+}} = -0.8 \text{ V}$, $E^{\circ}_{\text{Fe}^{3+}} = 0.77 \text{ V}$, $E^{\circ}_{\text{Cu}^{2+}} = 0.16 \text{ V}$.
\\ Since $E^{\circ}_{\text{Cell}} = E^{\circ}_{\text{Red}} + E^{\circ}_{\text{Ox}}$, 
\begin{equation}
    E^{\circ}_{\text{Cell}} = E^{\circ}_{\text{Red}} + 0.8 \text{ V}
    \label{eq:ecell}
\end{equation}
Rearranging equations (13) and (12) respectively from the lab manual\autocite{lab_manual},
\begin{equation}
    K_{12} = e^{\frac{zF}{RT}E^{\circ}_{\text{Cell}}} \text{ } 
    \label{eq:K12}
\end{equation}
\begin{equation}
    f_{12} = e^{\frac{(\ln{K_{12}})^2}{4\ln{(k_{11}k_{22} / Z_{12}^2)}}}
    \label{eq:f12}
\end{equation}
Equation (11) from the lab manual\autocite{lab_manual} gives:
\begin{equation}
    k_{\text{ET}} = k_{12} \approx \sqrt{k_{11}k_{22}K_{12}f_{12}}
    \label{eq:ket}
\end{equation}
Where $Z_{12} \approx 10^{11} \text{ M}^{-1} \text{s}^{-1}$, $T = 298.15\text{ K}$, $R = 8.3145\text{ Jmol}^{-1}\text{K}^{-1}$and $F = 96485.3399 \text{ Cmol}^{-1}$.
\\
\par The \textcolor{Lavender}{\mintinline{Python}{kET(k22, x_red_potential, z)}} Python function (see Appendix \ref{appx:part1_code} for full script) takes inputs for ($k_{22}$, $E^{\circ}_{\text{Red}}$, $z$), and substitutes above given values, equations (\ref{eq:ecell}), (\ref{eq:K12}), and (\ref{eq:f12}) to equation (\ref{eq:ket}), to return the $k_{\text{ET}}$ value.
% \input{marcus_theory_fn.txt}
This function was called for ($k_{22 \text{Fe}^{3+}}$, $E^{\circ}_{\text{Fe}^{3+}}$, $z = 1$), and ($k_{22, \text{Cu}^{2+}}$, $E^{\circ}_{\text{Cu}^{2+}}$, $z = 1$) to respectively return
\begin{equation*}
    \mathcolorbox{Lavender}{k_{\text{ET},\text{Fe}^{3+}} = \input{part1_q3_fe.txt} \text{ M}^{-1} \text{s}^{-1}}
\end{equation*}
\begin{equation*}
    \mathcolorbox{Lavender}{k_{\text{ET},\text{Cu}^{2+}} = \input{part1_q3_cu.txt} \text{ M}^{-1} \text{s}^{-1}}
\end{equation*}
